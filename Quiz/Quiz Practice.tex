\documentclass{article}
\usepackage{graphicx}
\usepackage{amsmath}

\title{\LaTeX \\Lab Quiz}
\author{Niloy Kumar Mondal}
\date{\today}

\begin{document}
   \maketitle
   \tableofcontents
   \pagebreak
   \section{Equation}
     \subsection{Lecture Codes}
       \subsubsection{Normal Distribution}
     
     \begin{equation}
         \label{eq:Normal_dist}
         \mathcal{N}( x_i;\mu,\sigma ) = \frac{1}{2\sqrt{\pi}\sigma}e^{-\frac{(x-\mu)^2}{2\sigma^2}}
    \end{equation}
    
    $ \mathcal{N}( x_i;\mu,\sigma ) = \frac{1}{2\sqrt{\pi}\sigma} e^{-\frac{(x-\mu)^2}{2\sigma^2}} $

    Look at equation \eqref{eq:Normal_dist} \\
    
      \subsubsection{Alignment Practice}
    This is another equation:
    \begin{align}
        E &= mc^2 \\
          &  e^{\frac{a^b}{c_d}} 
    \end{align}

    \subsubsection{Proper use of \emph{Bracket}}
    Fraction:
    $(\frac{\frac{a}{b}}{\frac{c}{d}})$ \\
    But we want to cover the total part.
      $\left( \frac{\frac{a}{b}}{\frac{c}{d}} \right)$ 

    \subsubsection{Piecewise function \& Simplification}
    \begin{align}
        |x| &= 
        \begin{cases}
            x &\text{if } x \geq 0 \\
            -x &\text{if } x < 0
        \end{cases} \\
        f(x) &= (x-1)(x+1) \nonumber \\
        &\qquad + e^x \nonumber \\
        &=x^2-1+e^x
    \end{align}

    \subsection{Online Solves}
      \subsubsection{\text{Vector}}
      \begin{equation*}
          \Vec{q}_x = \frac{2g_x}{k-1} \Vec{b}_n, \qquad
          \Vec{q}_y = \frac{2g_y}{m-1} \Vec{v}_n, \qquad
          \Vec{p}_{1m} = \Vec{t}_nd-g_x\Vec{b}_n-g_y\Vec{v}_n
      \end{equation*}

      \subsubsection{Nambu-Goto Action}
        \begin{equation}
            S = -T \int \sqrt{-\det(g_{ab})}\, d\tau \, d\sigma
        \end{equation}

       \subsubsection{Stirling number of second kind}
       \begin{equation}
           S(n,k)=\frac{1}{k!} \sum_{i=0}^{k} (-1)^{k-i} \binom{k}{i} i^n 
           = \sum_{i=0}^{k} \frac{(-1)^{k-i}\,i^n}{(k-i)!i!}
       \end{equation}

       \subsubsection{Complex piecewise function}
       \begin{align}
           F_c(x,y)=
           \begin{cases}
               \frac{\partial^2x^3y^x}{\partial x^2}\,+\,
               \frac{\partial^2\Gamma(x)\log(\tan y)}{\partial x \partial y} &\text{if x,y are real numbers} \\
               \displaystyle \lim_{Z \to e^{x^{2y}}}\sqrt{Z + \frac{a}{\sqrt{Z+\frac{1}{z+\dots}}}}
               &\text{otherwise}
           \end{cases}
       \end{align}


       \subsubsection{Eular Equation}
         \begin{equation}
         \label{Eular}
             e^{i\theta} = \cos\theta + i\sin\theta
         \end{equation}

         if we put $\theta = \frac{\pi}{2}$ in equation \eqref{Eular},we get the following:

         \begin{align*}
             e^{i\frac{\pi}{2}} &= \cos\frac{\pi}{2} + i \sin\frac{\pi}{2} \\
             &=0 + i . 1  \\
             &=i
         \end{align*}


       \section{Quiz}


    \begin{equation}
        |x| = 
        \begin{cases}
        \ x  & \text{ if } x \geq 0\\
        -x & \text{ if } x < 0 \\
           \ x  & \text{ if } x \geq 0\\
        \end{cases}
    \end{equation}
    
\end{document}